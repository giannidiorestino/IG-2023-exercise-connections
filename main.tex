 
\documentclass[xcolor=svgnames]{beamer}
%
\usetheme{Singapore}
\setbeamertemplate{navigation symbols}{}
\setbeamertemplate{footline}[frame number]

%
\setbeamercolor{block body}{bg=AliceBlue}

\usepackage[all]{xy}
\usepackage{accents}
\usepackage{amsmath,amssymb}
\usepackage{array}
\usepackage{bibentry}
\usepackage{bm}
\usepackage{cancel}
\usepackage{comment}
\usepackage{etex}
\usepackage{macros}
\usepackage{mathscinet}
\usepackage{mathtools}
\usepackage{rotating}
\usepackage{tikz}

\hypersetup{
    colorlinks=true,
    linkcolor=blue,
    filecolor=magenta,      
    urlcolor=cyan
    }
\renewcommand{\newblock}{\relax} %% ??????????

%
%% Gianni macros
%

\DeclareMathOperator{\M}{\mathcal M}
\DeclareMathOperator{\eDeriv}{D_{\text{e}}}
\DeclareMathOperator{\grad}{grad}
\DeclareMathOperator{\mDeriv}{D_{\text{m}}}

\newcommand{\Bspaceat}[1]{B_{#1}}
\newcommand{\Bspace}[1]{B_{#1}}
\newcommand{\Ccs}[1]{C_0\left(#1\right)}
\newcommand{\Cexp}[1]{C_0^{(\cosh-1)}\left(#1\right)}
\newcommand{\Cinfcs}[1]{C_0^\infty\left(#1\right)}
\newcommand{\Cinfp}[1]{C_{\mathrm{p}}^\infty\left(#1\right)}
\newcommand{\Derivby}[1]{\frac{\Deriv}{d#1}}
\newcommand{\Lexp}[1]{L^{(\cosh-1)}\left(#1\right)}
\newcommand{\LlogL}[1]{L^{(\cosh-1)_*}\left(#1\right)}
\newcommand{\TMaxexp}{\operatorname{T}\Maxexp}
\newcommand{\WCexp}[1]{C_0^{1,(\cosh-1)}\left( #1 \right)}
\newcommand{\Wexp}[1]{W^{1,(\cosh-1)}\left(#1\right)}
\newcommand{\WlogL}[1]{W^{1,(\cosh-1)_*}\left(#1\right)}
\newcommand{\bgamma}{{\bm \gamma}}
\newcommand{\bnabla}{{\bm\nabla}}
\newcommand{\condexpectat}[3]{{\Expectation}_{#1}\left[#2 \middle| #3\right]}
\newcommand{\displacement}{\operatorname{\mathbb S}}
\newcommand{\eBspace}[1]{B_{#1}}
\newcommand{\eDerivby}[1]{\frac{\eDeriv}{d#1}}
\newcommand{\ehessianat}[2]{\prescript{e}{}\Hessian_{#1}{#2}}
\newcommand{\expbundle}{S\Maxexp}
\newcommand{\fullbundleat}[1]{\prescript{1}{}S^1\maxexpat{#1}}
\newcommand{\gaussdensity}{\gamma}
\newcommand{\gaussint}[2]{\int{#1} \gaussdensity(#2) \ d#2 \ }
\newcommand{\hullof}[1]{\operatorname{hull}\left(#1\right)}
\newcommand{\mDerivby}[1]{\frac{\mDeriv}{d#1}}
\newcommand{\maxmix}[1]{\prescript{*}{}{\Maxexp\left(#1\right)}}
\newcommand{\mhessianat}[2]{\prescript{m}{}\Hessian_{#1}{#2}}
\newcommand{\mixbundleat}[1]{\prescript{*}{}S\maxexpat{#1}}
\newcommand{\mixbundle}{\prescript{*}{}S\Maxexp}
\newcommand{\mixfiberat}[2]{{}^*S_{#1}\maxexpat{#2}}
\newcommand{\model}{\mathcal M}
\newcommand{\opensimplexon}[1]{\mathcal P_>\left(#1\right)}
\newcommand{\preBspaceat}[1]{\prescript{*}{}B_{#1}}
\newcommand{\preBspace}[1]{\prescript{*}{}B_{#1}}
\newcommand{\rosso}[1]{\textcolor{red}{#1}}
\newcommand{\sdomainat}[1]{\sdomain_{#1}}
\newcommand{\sdomain}{\mathcal S}
\newcommand{\simplexon}[1]{\mathcal P\left(#1\right)}
\newcommand{\tensorat}[3]{\prescript{#1}{}S^{#2}\maxexpat{#3}}

\renewcommand{\emph}{\rosso}
\renewcommand{\transport}[2]{{\mathbb U} _ {#1} ^ {#2}}

%
%% end Gianni's macros
%

\title{\it IG 2023 Exercise CONNECTIONS}

\author[G Pistone]{\bf Giovanni Pistone}

\institute[CCA]{\includegraphics[height=4em]{pictures/deCastro-logo.pdf}}

\date{Revised \today}

\begin{document} 
% 
\begin{frame}\frametitle{DIMA UNIGE}  

\titlepage

\tiny 
web-page: \url{www.giannidiorestino.it} 

e-mail: \url{giovanni.pistone@carloalberto.org}

orcidID: 0000-0003-2841-788X

The author acknowledges the support of de Castro Statistics, DIMA U. Genoa, INdAM-GNAMPA

\nobibliography{tutto}%

\bibliographystyle{plain}

\end{frame}

\begin{frame}[plain,allowframebreaks]\small

Let us recall the basic definitions.

\begin{block}{}
The \emph{open probability simplex} on a finite sample space $X$ is
\begin{equation*}
   \maxexpat X = \setof{q \in \reals^X}{\sum_x q(x) = 1, q(x) > 0, x \in X} \ . 
\end{equation*}

The \emph{statistical bundle} is
\begin{equation*}
    \expbundleat X = \setof{(q,v)}{q \in \maxexpat X, \expectat q v = 0} \ .
\end{equation*}
The fiber at $q$ is the vector space
\begin{equation*}
    \expfiberat q X = \setof {v}{\expectat q v = 0} \ .
\end{equation*}

The \emph{covariance metric} is the mapping
\begin{equation*}
    S^2 \maxexpat X \ni (q,v,w) \mapsto \expectat q {vw} = \covat q u v \ .
\end{equation*}
\end{block}

\begin{block}{}
 The \emph{velocity} or \emph{Fisher's score} of a curve $t \mapsto q(t)$ is
 \begin{equation*}
     \velocity q(t) = \frac {\dot q(t)}{q(t)} = \derivby t \log q(t) \ .
 \end{equation*}
It holds
$t \mapsto (q(t),\velocity q(t)) \in \expbundleat X$
 \end{block}
 
\begin{block}{}
A \emph{section} or \emph{vector field} in $\expbundleat X$ is a function
\begin{equation*}
    A \colon \maxexpat X \ni q \mapsto A(q) \in \expfiberat q X \ .
\end{equation*}

The \emph{gradient in $\expbundleat X$ (w.r.t. the covariance metric} of a real function $\Phi \colon \maxexpat X \to \reals$ is a section $A = \grad \Phi$ such that 
\begin{equation*}
\derivby t \Phi(q(t)) = \scalarat {q(t)} {A(q(t))}{\velocity q(t)} \ .
\end{equation*}
\end{block}

\emph{Exercise 1} The gradient of the entropy $\entropyof q = \expectat q {- \log q}$ is
$\grad \entropyof q = -\log q - \entropyof q$.

\emph{Exercise 2} The gradient of the $f$-entropy $H_f(q) = \expectat q {f(q))}$ is $\grad H_f(q) = F(q) - \expectat q F(q)$ with $F(q) = f(q) + q f(q) = \derivby q q f(q)$. 
\end{frame}


\end{document}

%%% Local Variables:
%%% reftex-default-bibliography: ("/home/giannidiorestino/Dropbox/InProgress/tutto.bib")
%%% End:

